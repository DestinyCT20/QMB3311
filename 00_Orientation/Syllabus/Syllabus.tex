\documentclass[11pt]{paper}
\usepackage{geometry}
\usepackage{hyperref}
\usepackage{xcolor}
\usepackage{amsmath}
\geometry{
  top = 1in
  , bottom = 1in
  , left = 1in
  , right = 1in
  }
\hypersetup{
	colorlinks=true,
	linkcolor=blue,
	filecolor=magenta,
	urlcolor=cyan,
}

\begin{document}
\title{QMB 3311 Section 0001: Python for Business Analytics}
\author{University of Central Florida --- Department of Economics}

\maketitle
\hrulefill

\begin{table}[!htb]
    \begin{minipage}{.5\linewidth}
      \centering
 \begin{tabular}{| l | l |}\hline
 Course & QMB 3311-23Spring 0001 \\\hline
 Term & Spring 2023 \\\hline
 Meeting Time & TR 10:30am -- 11:50am\\\hline
 Location & BA1 212\\\hline
 Credit Hours & 3 \\\hline
\end{tabular}
    \end{minipage}%
    \begin{minipage}{.5\linewidth}
      \centering
\begin{tabular}{| l | l |}\hline
 Instructor & Joshua L. Eubanks \\\hline
 Office & BA2 302Y \\\hline
 Hours & TR 1:30pm -- 2:30pm or by appointment \\\hline
 Email & Email through Webcourses \\\hline
\end{tabular}
    \end{minipage} 
\end{table}




\section*{Important Dates}

\begin{itemize}
  \item \textbf{Midterm}: Feb 28, 2023 10:30am -- 11:50am
  \item \textbf{Final}: May 02, 2023 10:00am -- 12:50pm
\end{itemize}


\tableofcontents

\newpage

\section{Course Rules}

Please read these rules carefully.

\begin{enumerate}

\item Do not cheat. If a student is caught cheating, they will be punished to the fullest extent possible. Students should be familiar with Section \ref{sec:ai} of the syllabus.

\item I do not ``give'' grades, I merely record them. I am only able to change a grade if there is a clerical error. Any emails that fall into the category of ``academic panhandling'' (extra credit, ``grade bumping'', negotiations, etc.) will be ignored. I will not reevaluate coursework after the term is finished.

\item Make-up opportunities are not given. If a student misses a quiz or a homework, they can count it as a dropped assignment. If they miss the midterm exam, they must provide a valid excuse within a week of the missed midterm. If I deem the excuse valid, then their final exam will count in-place of the missed exam. 

\item Quizzes and exams can cover all the material (assignments, lectures, textbook, etc.) I have assigned. Questions about what questions, number of questions, etc. will not be discussed during class.

\item My office hours are to help students understand concepts that they may be struggling with. My hours do not replace attending class or reading the textbook.



\end{enumerate}


\section{Course Materials}
 
\subsection{Textbooks}

\begin{itemize}
\item Paul Gries, Jennifer Campbell, and Jason Montojo. \textit{Practical programming: an introduction to computer science using Python 3.6}. Pragmatic Bookshelf, 2017.

\item Matt Taddy. \textit{Business data science: Combining machine learning and economics to optimize, automate, and accelerate business decisions}. McGraw Hill Professional, 2019.

\item \texttt{Optional:}  Mortimer J Adler and Charles Van Doren. \textit{How to read a book: The classic guide to intelligent reading}. Simon and Schuster, 2014.

\end{itemize}

\subsection{Software}
\begin{itemize}
\item \texttt{Anaconda:} Anaconda is the most popular python distribution. It also installs many other modules by default, avoiding the requirement to install many of them manually. Additionally, it contains Spyder, a nice graphical user interface (GUI) that helps you handle python with ease.	
\item \texttt{GitHub Desktop:} GitHub is an excellent version control software used by many large companies. Think of it as a OneDrive or Google Drive for programmers. It provides a history of all the saved changes you made and allows for collaboration. GitHub can be executed from the command line, however, GitHub desktop removes the need to use the command line.
\end{itemize}

\section{Course Structure}

The tentative schedule is as follows:   

\begin{center}
\begin{tabular}{| l | l | l |}\hline
 Day & Material & Chapter(s) \\\hline 
 Jan 10 & Introduction & PP Ch. 1-2 \\
 Jan 12 & Using Functions & PP Ch. 3 \\
 Jan 17 & Designing Functions & PP Ch. 3 \\
 Jan 19 & Text and Strings & PP Ch. 4 ; BDS Ch. 8 \\
 Jan 24 & Boolean Variables & PP Ch. 5\\
 Jan 26 & Lists & PP Ch. 8 \\
 Jan 31 & Loops & PP Ch. 9\\
 Feb 02 & Python Modules & PP Ch. 6 \\
 Feb 07 & Using Modules & PP Ch. 6 \\
 Feb 09 & Designing Modules & PP Ch. 6 \\
 Feb 14 & Methods & PP Ch. 7\\
 Feb 16 & File IO & PP Ch. 10 \\
 Feb 21 & Reading Files & PP Ch. 10 \\
 Feb 23 & Sets and Tuples & PP Ch. 11 \\
 Feb 28 & Midterm Exam & --- \\
 Mar 02 & Algorithms & PP Ch. 12 \\
 Mar 07 & Solving Equations & --- \\
 Mar 09 & Searching & PP Ch. 13 \\
 Mar 14 & Spring Break: No Classes & --- \\
 Mar 16 & Spring Break: No Classes & --- \\
 Mar 21 & Sorting & PP Ch. 13\\
 Mar 23 & Classification & BDS Ch. 4 \\
 Mar 28 & Optimization & BDS Ch. 2-3 \\
 Mar 30 & Nonparametrics & BDS Ch. 9 \\
 Apr 04 & TBD & --- \\
 Apr 06 & Dictionaries & PP Ch. 11 \\
 Apr 11 & Creating Databases & PP Ch. 17 \\
 Apr 13 & Using Databases & PP Ch. 17 \\
 Apr 18 & Using Databases & PP Ch. 17 \\
 Apr 20 & Testing & PP Ch. 15 \\\hline
\end{tabular}
\end{center}

\section{Course Description and Learning Objectives}

At the end of the course, a student should be able to solve quantitative problems using python. Those interested in a career in business analytics will benefit from this course. After completing the course, a student should be able to:

\begin{itemize}
\item Understand data types and basic operations
\item Use existing python modules to solve problems
\item Create new python modules and functions
\item Understand linear alegrbra and implement it via python
\item Manage and analyze data to solve business problems
\item Understand the distinction between written calculations and computer calculations 
\end{itemize}

For the official course description please refer to the \href{http://catalog.ucf.edu/search_advanced.php?catoid=14}{Undergraduate Catalog}.

\section{Grading and Assignments}

Your overall grade can be calculated using this formula:

  \begin{equation*}
    \text{Overall Score} = \max
    \begin{cases}
      0.1 \text{(Top 6 Quizzes)} + 0.3 \text{(Top 6 Assignments)} + 0.3 \text{(Midterm)} + 0.3 \text{(Final)} \\
      0.1 \text{(Top 6 Quizzes)} + 0.3 \text{(Top 6 Assignments)} + 0.15 \text{(Midterm)} + 0.45 \text{(Final)}
    \end{cases}
  \end{equation*}

Using this formula, you will be able to calculate your grade without my help.

\subsection{Grading Scale}

I have a strict grading policy: your course grade will be based upon the extent to which you meet \textit{my} expectations in the course. I do not curve the homework, quizzes, or exams.

\begin{flushleft}
\begin{tabular}{ l  l  l }\hline
 Percent & Grade & \\\hline 
 100-93 &  A & outstanding performance, demonstrates a complete grasp of the material\\
 92.9-90 & A- & excellent performance, only minor errors throughout\\
 89.9-87 & B+ & very good performance, several minor errors or a few more substantial ones\\
 86.9-83 & B & good performance, more of both types of the aforementioned errors\\
 82.9-80 & B- & pretty good performance, substantial errors becoming more common\\
 79.9-77 & C+ & fair performance, frequent errors of substance \\
 76.9-73 & C & fair performance, even more frequent errors of substance\\
 72.9-70 & C- & weak performance, major problems appear\\
 69.9-67 & D+ & weak performance, major problems appear more regularly\\
 66.9-63 & D & poor performance, major problems are common\\
 62.9-60 & D- & poor performance, a lack of understanding demonstrated regularly \\
 $\leq$59.9 & F & ---\\\hline
\end{tabular}
\end{flushleft}


\subsection{Assignments and Assessments}

These are the ways you will be evaluated throughout the course.

\subsubsection{Quizzes}
There will be at least 6 quizzes. The quizzes open-book and open-notes, cover any material presented, and will be approximately 15 minutes in length. It can be at the beginning or end of the class, so be sure to read the textbook in preparation.

\subsubsection{Assignments}
You can expect 8 assignments throughout the semester, but I only count the top 6. Be sure to complete all the assignments as future assessments may require you to reference previous homework.

\subsubsection{Midterm}
This will be an exam on February 28, 2023. The exam is written to be completed within 1 hour and 15 minutes. The exam will be digital and focuses on quantitative problem solving using python. It is open note, so utilize any resourses you have.

\subsubsection{Final Exam}
According to UCF policy, all courses must have a final examination or assessment. This exam is cumulative and is written to be completed within 2 hours and 50 minutes. The rules will be the same as the midterm.

\section{Reporting}
Grades will be reported via Webcourses to follow student data classification and security standards. I cannot discuss your grades with anyone else. I will make comments on your submissions, but will not display the point deductions.

\section{Policy Statements}
\subsection{Academic Integrity} \label{sec:ai}
The Center for Academic Integrity (CAI) defines academic integrity as a commitment, even in the face of adversity, to five fundamental values: honesty, trust, fairness, respect, and responsibility. From these values flow principles of behavior that enable academic communities to translate ideals into action.\footnote{\url{https://academicintegrity.org/}}\\

The \href{https://osrr.sdes.ucf.edu/}{Office of Student Rights and Responsibilities}\footnote{Located in Ferrell Commons, Room 227} will be notified of any instance of academic misconduct that has occurred inside or outside of the classroom. Students are encouraged to read the \href{https://goldenrule.sdes.ucf.edu/}{Golden Rule Student Handbook}.\\

Students should familiarize themselves with UCF’s Rules of Conduct. According to Section 1, ``Academic Misconduct,'' students are prohibited from engaging in
\begin{enumerate}
\item Unauthorized assistance: Using or attempting to use unauthorized materials, information or study aids in any academic exercise unless specifically authorized by the instructor of record. The unauthorized possession of examination or course-related material also constitutes cheating.
\item Communication to another through written,visual,electronic, or oral means:The presentation of material which has not been studied or learned, but rather was obtained through someone else’s efforts and used as part of an examination, course assignment, or project.
\item Commercial Use of Academic Material: Selling of course material to another person, student, and/or uploading course material to a third-party vendor without authorization or without the express written permission of the university and the instructor. Course materials include but are not limited to class notes, Instructor’s PowerPoints, course syllabi, tests, quizzes, labs, instruction sheets, homework, study guides, handouts, etc.
\item Falsifying or misrepresenting the student’s own academic work.
\item Plagiarism: Using or appropriating another’s work without any indication of the source, thereby attempting to convey the impression that such work is the student’s own.
\item Multiple Submissions: Submitting the same academic work for credit more than once without the express written permission of the instructor.
\item Helping another violate academic behavior standards.
\end{enumerate}
For more information about plagiarism and misuse of sources, see ``Defining and Avoiding Plagiarism: The WPA Statement on Best Practices.''\\


UCF faculty members strive to provide a quality education, and so seek to prevent unethical behavior and when necessary, respond to infringements of academic integrity. Penalties can include a failing grade in an assignment or course, suspension/expulsion from the university, and/or a ``Z'' designation\footnote{More information on Z designation \href{https://goldenrule.sdes.ucf.edu/zgrade}{here}.} on a students official transcript designating academic dishonesty. 

\subsection{Active Duty Military}
Students under active duty in the military will be accommodated as much as possible. Please see me  prior to scheduled military obligations if this applies to you.
\subsection{Attendance/Late Policies}
Late work will not be accepted. Attendance will not be recorded, however, the quizzes administered will not be announced and lecture notes will not be uploaded. If you show up over 20 minutes after we have started the midterm or final exam, you will \textbf{NOT} be able to take the exam.
\subsection{Emergency Procedure and Campus Safety}
Be aware of your surroundings and be familiar with the necessary actions to take in the event of an emergency. In case of an emergency, dial 911 for assistance. All classrooms contain an emergency procedure guide and is available \href{http://emergency.ucf.edu/emergency_guide.html}{online}. I advise signing up for text alerts from UCF if not already registered. Steps are below:
\begin{itemize}
	\item Log in to myUCF
	\item Click the `Student Self Service' tab
	\item Click the `Personal Information' tab
	\item Click the `UCF Alert' tab
\end{itemize} 

If there is a medical emergency during class, students may need to access a
first-aid kit or AED (Automated External Defibrillator). Here is the \href{http://www.ehs.ucf.edu/AEDlocations-UCF}{link} to learn where those are located.\\

To learn about how to manage an active-shooter situation on campus or elsewhere, consider viewing this \href{https://www.youtube.com/watch?v=NIKYajEx4pk&feature=youtu.be}{video}.\\

Students with special needs related to emergency situations should speak with me outside of class.

\subsection{Make-up Exams and Assignments}
Per university policy, students may only turn in make-up work (or and equivalent, alternate assignment) for \textbf{university-sponsored events, religious observances, or legal obligations (i.e. jury duty)}. In these instances, students are excused without penalty.\\

Students who know they will be absent due to a religious observance must notify me at the beginning of the semester so that make-up work can be arranged. For more information, please refer to the \href{https://regulations.ucf.edu/docs/notices/5.020ReligiousObservancesNEW_Oct09_000.pdf}{policy}.

\subsection{Revisions to the Syllabus}
I reserve the rights to make changes to the syllabus as we progress through the semester. I will post an announcement declaring any major changes to the syllabus through Webcourses.

\subsection{Student Academic Activity Policy}
As of Fall 2014, all faculty members are required to document the student's academic activity at the beginning of the course. How I am documenting it in this course is the first quiz. Please complete the quiz by the end of the first week, or you may delay/lose your financial aid. 
\subsection{Student Accessibility Services}
The University of Central Florida is committed to providing access and inclusion for all persons with disabilities. Students with disabilities who need disability-related access in this course should contact the lecturer as soon as possible. Students should also connect with \href{https://sas.sdes.ucf.edu}{Student Accessibility Services} located at Ferrell Commons room 185, by \href{mailto:emailsas@ucf.edu}{email}, or phone 407-823-2371. Through Student Accessibility Services, a Course Accessibility Letter may be created and sent to professors, which informs faculty of potential access and accommodations that might be reasonable. Determining reasonable access and accommodations requires consideration of the course design, course learning objectives and the individual academic and course barriers experienced by the student.

\end{document}