\documentclass[11pt]{exam}


\begin{document}

\texttt{Assignment 1 --- QMB 3311 ---  Spring 2025  --- Due: January 13, 2025 @ 23:59}


\subsection*{GitHub Instructions:}

\begin{enumerate}
    \item Create a GitHub account
    \item In your account, create a repository named: \texttt{LastNameFirstName\_QMB3311}. For example, my repository would be \texttt{EubanksJoshua\_QMB3311}
    \item Add me as a collaborator (Username: \texttt{JoshuaEubanksUCF})
    \item Create a folder called \texttt{Assignment\_01}
    \item Create a python script called \texttt{Assignment\_01.py}

\end{enumerate}

\subsection*{Instructions:}

  \begin{questions}

\question Write a script that follows these steps to create a ``magic trick'' from Dr. Dadkhah's book \textit{Foundations of Mathematical and Computational Economics}.


	\begin{parts}
\part Pick a number between one and nine (the number of movies you'd like to see in a week or the number of times you prefer to eat out)

\part Multiply the number by 2

\part Add 5

\part Multiply the result by 50

\part If you had already had your birthday this year, add 1750; or else add 1749

\part Now, what year is this? 2006? 2007? 2008? Whatever it is, add the last digit of the year, say 6 if it is 2006 or 12 if it is 2012.

\part Now, subtract the year you were born (all four digits, e.g., 1984)

\part You should have a three-digit number. The first digit is your original number; the next two numbers are your age!
        \end{parts}


\question Save your script and commit the changes to your repository. I will grade the last commit before the due date.

    \end{questions}

\end{document}