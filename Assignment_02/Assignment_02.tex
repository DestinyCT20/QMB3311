\documentclass[11pt]{exam}


\begin{document}

\texttt{Assignment 2 --- QMB 3311 ---  Spring 2025  --- Due: January 20, 2025 @ 23:59}


\subsection*{Group Repo Instructions:}

\begin{enumerate}
    \item Join a group in Webcourses.
    \item One person from your group: a GitHub repository called \texttt{QMB3311\_GroupXX\_LastName1\_LastName2} where the name ordering is in alphabetical order. For example if Dr. Paarsch and I were in group 5, I would create a repository called \texttt{QMB3311\_Group05\_Eubanks\_Paarsch}.
    \item Invite me as a collaborator (\texttt{JoshuaEubanksUCF})
\end{enumerate}

\subsection*{Assignment Instructions:}

Complete this assignment within the space on your group's GitHub repo in a folder called \texttt{Assignment\_02}. In this folder, save your answers to a file called \texttt{my\_functions.py}, following the sample script in the folder \texttt{Assignment\_02} in the course repository. When you are finished, submit your files to your repository.

    \begin{questions}

\question Follow the function design recipe to define functions for all of the following exercises. for each function, create three examples to test your functions. Record the definitions in the sample script \texttt{my\_functions.py}


	\begin{parts}
\part Write a python function \texttt{present\_value()} that will calculate the present value of a future cash flow. It should have 3 arguments, the dollar amount of the cash flow, the discount rate, and the number of years that the cash flow is to be received in the future.

\part Write a python function \texttt{future\_value()} that will calculate the \textit{future} value of a \textit{present} cash flow. It should have 3 arguments, the dollar amount of the cash flow, the discount rate, and the number of years that the cash flow is to be invested for the future.

\part Write a python function \texttt{total\_revenue()} that will calculate the revenue earned by a firm selling a product at a fixed price. The first argument should be the number of units sold, the second argument should be the price

\part Write a python function \texttt{total\_cost()} that will calculate the total cost incurred by a firm to produce a product. The first argument should be the the number of units produced, the second should be the fixed costs, third argument is a constant multiplied by the square of the number of units sold. That is, the cost function could be written as $TC(q,a,b) = aq^{2}+b$ where $b$ is the fixed cost, $q$ is the quantity produced, and $a$ is the number multiplied by the square of the number of units

\part Write a python function \texttt{CESutility()} that will calculate the value of the Constant Elasticity of Substitution utility function $u(x,y;r) = \left(x^{r}+y^{r}\right)^{\frac{1}{r}}$, which measures the theoretical degree of satisfaction a consumer may get from two goods. The first two arguments $x,y$ are the two goods consumed and the third is $r$, which is a parameter that represents the degree to which the goods are complements or substitutes.
        \end{parts}

    \end{questions}

\end{document}