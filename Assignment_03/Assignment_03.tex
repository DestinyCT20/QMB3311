\documentclass[11pt]{exam}


\begin{document}

\texttt{Assignment 3 --- QMB 3311 ---  Spring 2024  --- Due: February 3, 2025}

\subsection*{Instructions:}

Complete this assignment within the space on your group's GitHub repo in a folder called \texttt{Assignment\_03}. In this folder, save your answers to a file called \texttt{my\_A3\_functions.py}, following the sample script in the folder \texttt{Assignment\_03} in the course repository. When you are finished, commit and push your changes to the group repository.

    \begin{questions}

\question Follow the function design recipe to define functions for all of the following exercises. for each function, create three examples to test your functions. Record the definitions in the sample script \texttt{my\_A3\_functions.py}


	\begin{parts}
\part For Assignment 2, you wrote a function \texttt{CESutility()} that calculated the value of the Constant Elasticity of Substitution utility function $u(x; y; r) = \left(x^{r} + y^{r}\right)^{\frac{1}{r}}$ . In this function, the first two arguments are $x$ and $y$, respectively, and the third is $r$. Write an augmented version of the function called \texttt{CESutility\_valid()} that returns the same value as \texttt{CESutility()} when $x$ and $y$ are non-negative numbers and $r$ is strictly positive but returns the value \texttt{None} otherwise. For each case of negative numbers, make your function print a message that tells the user what is wrong with the inputs.

\part Extend the above function with another \textit{wrapper} function \texttt{CESutility\_in\_budget(x, y, r, p\_x, p\_y, w)} that evaluates \texttt{CESutility\_valid()} when the consumer's choice of goods $x$ and $y$ are in budget and returns \texttt{None} otherwise. That is, given prices $p_{x}$ and $p_{y}$, the consumer's basket of goods should cost no more than their wealth $w$: $w \geq p_{x}x+p_{y}y$. The function should also return \texttt{None} if any of the prices are negative or if $r$ is not positive.

\part Write a python function \texttt{logit()} that will calculate the logit link function

$$\ell(x;\beta_{0},\beta_{1}) = Prob(y=1 | x) = \frac{e^{\beta_{0}+x\beta_{1}}}{1 + e^{\beta_{0}+x\beta_{1}}}$$

The first argument is $x$ and the last two are $\beta_{0}$ and $\beta_{1}$. For your examples, you may use the fact that $e^{\ln a} = a $ to create examples that evaluate to fractions.

\part The likelihood function of the logistic regression model is used to estimate coefficients in logistic regression. Logistic regression is used to model binary events, i.e. whether or not an event occurred. For each observation $i$, the observation $y_{i}$ equals 1 if the event occurred and 0 if it did not. Build on the function from above and write a python function \texttt{logit\_like()} that calculates the log-likelihood of observation ($y_i$; $x_i$). That is, it returns the log of the function $\ell(x;\beta_{0},\beta_{1})$ if $y_{i} = 1$ or the log of the function $1-\ell(x;\beta_{0},\beta_{1})$ if $y_{i} = 0$. This function will have four arguments $(y_{i}, x_{i}; \beta_{0},\beta_{1})$ in that order.



        \end{parts}

    \end{questions}

\end{document}