\documentclass[11pt]{exam}

\usepackage{amsmath}

\begin{document}

\texttt{Assignment 5 --- QMB 3311 ---  Spring 2023  --- Due: March 26, 2023}

\subsection*{Instructions:}

Complete this assignment within the space on your \textit{private} GitHub repo in a folder called \texttt{Assignment\_05}. In this folder, save your answers to a file called \texttt{my\_CES\_A5.py} and \texttt{my\_logit\_A5.py} in the course repository. When you are finished, submit your files to your repository and upload the link to Webcourses.

For all of the exercises, use your examples to test the functions you defined. Since the examples are all contained within the docstrings of your functions, you can use the
\texttt{doctest.testmod()} function within the doctest module to test your functions automatically.

Don't worry about false alarms: if there are some ``failures'' that are only different in the smaller decimal places, then your function is good enough. It is much more important that your function runs without throwing an error.

    \begin{questions}

\question Create the following functions from your previous assignments into a two different modules:

\begin{itemize}
    \item For Module \texttt{my\_CES\_A5.py}, copy these previous functions:

    \begin{itemize}
      \item \texttt{CES\_utility()} from Assignment 2.
      \item \texttt{CES\_utility\_valid()} from Assignment 3.
      \item \texttt{CES\_utility\_in\_budget()} from Assignment 3.
    \end{itemize} 

    \item For Module \texttt{my\_logit\_A5.py}, copy these previous functions:

    \begin{itemize}
      \item \texttt{logit()} from Assignment 3.
      \item \texttt{logit\_like()} from Assignment 3.
      \item \texttt{logit\_like\_sum()} from Assignment 4.
    \end{itemize}  
\end{itemize}

\question Follow the function design recipe to define functions for all of the following parts. For each
function, create three examples to test your functions.

	\begin{parts}
\part Write a \textit{helper} function \texttt{logit\_d\_i(x\_i,k)} that helps you calculate the term $d_{i}$ in the gradient vector. The formula for \texttt{logit\_di()} is 

$$d_{i} = \begin{cases}
1, & \text{if } k = 0,\\
x_{i}, & \text{if } k = 1,\\
\text{undefined} & \text{ otherwise.}
\end{cases}$$

where $x_{i}$ is a number. It will be a single observation of a list of covariates \texttt{x} when \texttt{logit\_d\_i()} is used in another function. Add this function your module \texttt{my\_logit\_A5.py}.

\part Write another helper function \texttt{logit\_dLi\_dbk(y\_i,x\_i, beta\_0, beta\_1)} that helps you calculate an individual term in the sum of the gradient vector. The formula for \texttt{logit\_dLi\_dbk()} is
is

\begin{equation*}
\frac{\partial}{\partial \beta_{k}} L_{i}(y_{i},x_{i};\beta_{0},\beta_{1}) = 
\begin{cases}
    d_{i}(1-\ell(x_{i};\beta_{0},\beta_{1})), & \text{if } y_{i} = 1,\\ 
    d_{i}(-\ell(x_{i};\beta_{0},\beta_{1})), & \text{if } y_{i} = 0,\\
    \text{undefined} & \text{otherwise}
\end{cases}
\end{equation*}

where $y_{i}$ and $x_{i}$ is a single pair of observations from a list of observations in \texttt{y} and \texttt{x} that can be used in another function. Add this function your module \texttt{my\_logit\_A5.py}.

\part Write a function \texttt{CESdemand\_calc(r, p\_x, p\_x, w)} that returns a list of two values \texttt{[x\_star,y\_star]} that achieve the maximum value of \texttt{CES\_utility()}, subject to the budget constraint that the consumer's basket of goods should cost no more than their wealth $w$: $p_{x}x+p_{y}y \leq w$.
That is, given $p_{x}$ and $p_{y}$, these values maximize the function \texttt{CESutility\_in\_budget}, without
returning a value of \texttt{None}. A senior analyst used calculus to find the optimal values of $x^{*}$ and
$y^{*}$:

$$x^{*} = \frac{p_{x}^{\frac{1}{r-1}}}{p_{x}^{\frac{r}{r-1}} + p_{y}^{\frac{r}{r-1}}} \times w \text{ and } y^{*} = \frac{p_{y}^{\frac{1}{r-1}}}{p_{x}^{\frac{r}{r-1}} + p_{y}^{\frac{r}{r-1}}} \times w$$


Add this function to your module \texttt{my\_CES\_A5.py}.

\part Now write a function that finds values of \texttt{x} and \texttt{y} that maximize \texttt{CESutility\_in\_budget(x, y, r, p\_x, p\_y, w)} for given \texttt{r, p\_x, p\_y,} and \texttt{w}. Write a function definition \texttt{max\_CES\_xy(x\_min, x\_max, y\_min, y\_max, step, r, p\_x, p\_y, w)} as follows:

\begin{enumerate}
    \item Find the values of x and y by evaluating \texttt{CESutility\_in\_budget(x, y, r, p\_x, p\_y, w)} over every combination of of $(x, y)$ in two lists.
    \item Create lists \texttt{x\_list} and \texttt{y\_list} from ranges $x = x^{min}, \dotsc, x^{max}$ and $y = y^{min}, \dotsc,  y^{max}$, where the neighboring values of $x$ or $y$ are separated by distance \texttt{step}. The \texttt{np.arange()} function is useful for this.
    \item Initialize the maximized value with \texttt{max\_CES = float(``-inf'')}.
    \item Loop over the index numbers \texttt{i} and \texttt{j} corresponding to lists \texttt{x\_list} and \texttt{y\_list}.
    \begin{enumerate}
        \item For each pair of \texttt{i} and \texttt{j}, extract the value \texttt{x\_list[i]} and \texttt{y\_list[j]}
        \item For each pair of \texttt{i} and \texttt{j}, evaluate \texttt{CESutility\_in\_budget(x, y, r, p\_x,p\_y, w)}
        \item If the value is higher than \texttt{max\_CES}, record the new \texttt{i\_max = i} and \texttt{j\_max = j} and update the newest value of the \texttt{max\_CES}. If \texttt{CESutility\_in\_budget} returns \texttt{None}, make no changes and move on to the next values.  
    \end{enumerate}

    \item After the loops, return \texttt{[ x\_list[i\_max], y\_list[j\_max] ]}.

    \item Verify that the result matches the values from the previous question (up to accuracy \texttt{step}). You can use the same test cases as you did in the previous question. Add this function to your module \texttt{my\_CES\_A5.py}
\end{enumerate}

        \end{parts}

    \end{questions}

\end{document}